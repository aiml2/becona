\documentclass{beamer}

\usepackage[utf8]{inputenc}
\usepackage{default}
\usepackage{amsmath}
\usepackage{tikz}

\usetheme{CambridgeUS}

\title{Machine Learning @ Becona}
\subtitle{Proof of Concept Image Classification for Concrete Casting Equipment}

\author{Dieter Castel}
\date{\today}

\begin{document}

\begin{frame}
  \titlepage
\end{frame}

\begin{frame}{overview}
\tableofcontents
\end{frame}

\section{Problem}
\begin{frame}{Problem Definition}
  \begin{itemize}
	    \item Automated Sorting of Concrete Casting Equipment 
	    \begin{itemize}
	      \item Exploratory Proof of Concept 
	      \item Focused on Object Recognition Software System
	      \item Educational project for myself 
	    \end{itemize}
	    \item \textbf{Are Convolutional Neural Networks a valid approach for recognizing the Becona rental items?}
  \end{itemize}
\end{frame}

\begin{frame}{Items}
  \begin{itemize}
	    \item About 30 items exist.
	    \item I focused on a subset of 6: \\
	    \begin{tabular}{ccc}
	      Id & Name & \# samples \\ \hline
	      1 & Spanklem & 223 \\
	      2.0 & Vleugelmoer Opleg Recht - Oud rond twee gaten & 208 \\
	      2.1 & Vleugelmoer Opleg Recht - Nieuw geen gaten  & 244 \\
	      3 & Vleugelmoer Opleg Rond & 238 \\
	      4.0 & Variable Spanklem - Kort & 270 \\
	      4.1 & Variable Spanklem - Lang & 251  \\
	    \end{tabular}
	    \item Of which 2.x and 4.x have very high visual similarity
  \end{itemize}
\end{frame}

\section{Approach}

\begin{frame}
\begin{itemize}
 \item Training a Neural Network from scratch needs millions of samples 
 \item In two afternoons I took merely 1434 pictures
 \item \textbf{solution}: stand on the shoulder of giants $\rightarrow$ \textbf{transfer learning}
\end{itemize}
\end{frame}

\begin{frame}{Neural Network structure (simplified)}
\begin{figure}
\def\layersep{2.5cm}

\begin{tikzpicture}[shorten >=1pt,->,draw=black!50, node distance=\layersep]
    \tikzstyle{every pin edge}=[-,shorten <=1pt]
    \tikzstyle{neuron}=[circle,fill=black!25,minimum size=17pt,inner sep=0pt]
    \tikzstyle{input neuron}=[neuron];
    \tikzstyle{output neuron}=[neuron];
    \tikzstyle{hidden neuron}=[neuron];
    \tikzstyle{annot} = [text width=4em, text centered]

    % Draw the input layer nodes
    \foreach \id / \x / \y / \rgb / \color in {1/0/0/R/red!50,2/0/0/G/green!30,3/0/0/B/blue!80,4/1/0/R/red}
	\node[input neuron, fill=\color, pin=left:{\rgb-Pixel (\x,\y)}] (I-\id) at (0,-\id) {};

    % Draw the hidden layer nodes
    \foreach \name / \y in {1,...,5}
        \path[yshift=0.5cm]
            node[hidden neuron] (H-\name) at (\layersep,-\y cm) {};

    \foreach \y / \name in {1/1,2/2.0,3/2.1,4/3,5/4.0,6/4.1}
        \path[yshift=0.5cm]
            node[output neuron, pin=right:{$P(\name)$}] (O-\y) at (2*\layersep,-\y cm) {};

    % Connect every node in the input layer with every node in the
    % hidden layer.
    \foreach \source in {1,...,4}
        \foreach \dest in {1,...,5}
            \path (I-\source) edge (H-\dest);

    % Connect every node in the hidden layer with every node in the
    % output layer.
    \foreach \source in {1,...,5}
        \foreach \dest in {1,...,6}
            \path (H-\source) edge (O-\dest);

    % Annotate the layers
    \node[annot,above of=H-1, node distance=1cm] (hl) {Hidden layers};
    \node[annot,left of=hl] {Input layer}; % 299x299 RGB image
    \node[annot,right of=hl] {Output layer};
    \node[annot,below of=I-2] {$\vdots$};
    \node[annot,below left of=I-3] {$\vdots$};
\end{tikzpicture} 
\end{figure}
\end{frame}

\section{Results}
\section{Suggestions}
\section{Q\&A}
\begin{frame}
  \begin{itemize}
	    \item How does the system compare to human classification?
	    \item Speed, cost, effort, \emph{accuracy}, FP vs FN
	    \item TEST
  \end{itemize}
\end{frame}


\end{document}
